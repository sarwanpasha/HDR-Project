%%%%%%%%%%%%%%%%%%%%%%%%%%%%%%%%%%%%%%%%%
% University/School Laboratory Report
% LaTeX Template
% Version 3.0 (4/2/13)
%
% This template has been downloaded from:
% http://www.LaTeXTemplates.com
%
% Original author:
% Linux and Unix Users Group at Virginia Tech Wiki
% (https://vtluug.org/wiki/Example_LaTeX_chem_lab_report)
%
% License:
% CC BY-NC-SA 3.0 (http://creativecommons.org/licenses/by-nc-sa/3.0/)
%
%%%%%%%%%%%%%%%%%%%%%%%%%%%%%%%%%%%%%%%%%

%----------------------------------------------------------------------------------------
%	PACKAGES AND DOCUMENT CONFIGURATIONS
%----------------------------------------------------------------------------------------

\documentclass[a4paper]{article}

\usepackage{mhchem} % Package for chemical equation typesetting
\usepackage{siunitx} % Provides the \SI{}{} command for typesetting SI units

\usepackage{graphicx} % Required for the inclusion of images
\usepackage{url}

\setlength\parindent{0pt} % Removes all indentation from paragraphs

\renewcommand{\labelenumi}{\alph{enumi}.} % Make numbering in the enumerate environment by letter rather than number (e.g. section 6)

%\usepackage{times} % Uncomment to use the Times New Roman font

%-------------------------------------------------------------------------------
%	DOCUMENT INFORMATION
%-------------------------------------------------------------------------------

\title{Project Definition: \\ HDR imaging\\ Multimedia Communications AS 2015} %Title

\author{Taivo Pungas \\ 15-928-336} % Author name

\date{\today} % Date for the report

\begin{document}

\maketitle % Insert the title, author and date

% If you wish to include an abstract, uncomment the lines below
% \begin{abstract}
% Abstract text
% \end{abstract}

%-------------------------------------------------------------------------------
%	SECTION 1
%-------------------------------------------------------------------------------

\section{Motivation}
Most classified ads -- ranging from real estate to used guitars -- contain images. In most cases these images are taken by non-experts, and even when a professional photographer is used, photos might not reflect the advertised object accurately; one possible cause is the low dynamic range of the images.

Assuming the look and feel of images play an important part in buying decisions, we naturally get to the questions: do scenes or objects depicted in HDR images look better compared to LDR images? Which tone mapping operator (TMO) gives the best result?

%-------------------------------------------------------------------------------
%	SECTION 2
%-------------------------------------------------------------------------------

\section{Approach}

I will:
\begin{enumerate}
	\item Create HDR images by combining multiple images taken at different exposures and tone-map the images using a Matlab toolbox\footnote{\url{https://github.com/banterle/HDR_Toolbox}} from \cite{Banterle:2011}.
	\item In a survey\footnote{Potentially using the Amazon Mechanical Turk Sentiment App: \url{https://requester.mturk.com/create/sentiment/about}}, ask users how pleasant and realistic the images are.
	\item Automatically extract aesthetic quality scores from the image, using the method from \cite{aydin2015automated}.
	\item Compare the results of LDR images and different TMOs for HDR images.
\end{enumerate}


%-------------------------------------------------------------------------------
%	SECTION 3
%-------------------------------------------------------------------------------

\section{Expected Outcome}
I will rank different TMOs and compare them with LDR images. I will find out whether this ranking depends on the object (person vs dog vs landscape), lighting conditions and possibly other factors.


%-------------------------------------------------------------------------------
%	BIBLIOGRAPHY
%-------------------------------------------------------------------------------

\bibliographystyle{unsrt}

\bibliography{report}

%-------------------------------------------------------------------------------


\end{document}